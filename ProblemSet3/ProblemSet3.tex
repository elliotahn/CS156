\documentclass[12pt]{article}
\usepackage{amsmath, amssymb, amsthm}
\usepackage[margin=1in]{geometry}
\usepackage{fancyhdr, enumitem}
\usepackage{graphicx, tikz}
\usepackage[labelfont=bf]{caption}

\pagestyle{fancy}
\fancyhf{}
\chead{PROBLEM SET 3}
\rhead{Elliot Ahn}
\lhead{Machine Learning}
\rfoot{\thepage}

\setlength{\headheight}{15pt}
\renewcommand{\footrulewidth}{0.5pt}

\begin{document}

\begin{enumerate}[leftmargin=*]
\item (b)
\begin{align*}
2 M e^{- 2 \epsilon^2 N} &\leq \delta \\
- 2 \epsilon^2 N &\leq \ln \frac{\delta}{2 M} \\
N & \geq \frac{1}{2 \epsilon^2} \ln \frac{2 M}{\delta}
\end{align*}
For $\epsilon = 0.05$, $M = 1$, and $\delta = 0.03$, we get $N \geq 840$.
\item (c) Plug in $M = 10$ instead and we get $N \geq 1300$.
\item (d) Plug in $M = 100$ and we get $N \geq 1761$.
\item (b) Break point for perceptron is $n + 2$ where $n$ is the number of dimensions. So this time we get $5$.
\item (b) The growth function is either polynomial in $N$ or $2^N$.
\item (c) $h$ can only shatter 4 points because we can only have 4 sign switches while allowing for the smallest point to be positive. So the smallest break point is 5.
\item (c) We count the number of ``switches'' from negative points to positive points and from positive points to negative points. We can have 0 to 4 switch points, so the growth function is (using Pascal's rule)
\begin{align*}
m_{\mathcal H} &= \binom{N}{0} + \binom{N}{1} + \binom{N}{2} + \binom{N}{3} + \binom{N}{4} \\
&= 1 + \binom{N + 1}{2} + \binom{N + 1}{4}
\end{align*}
\item (d) We need to find $N$ where
\[ \sum_{k = 0}^{2M} \binom{N}{k} < 2^N. \]
I just let $M = 12$ and on a computer found that the smallest such $N$ is $N = 25$. So our break point is $2 M + 1$.
\item (d) Just draw a regular polygon of $N$ sides and think through it.
\item (b) Perform a transformation and order the points by
\[ (x, y) \to r \equiv \sqrt{x^2 + y^2}. \]
This is equivalent to the interval problem where given an interval, a point is $+1$ if it's in the interval and $-1$ otherwise. This gives
\[ m_{\mathcal H} = \sum_{k=0}^2 \binom{N}{k} = 1 + \binom{N + 1}{2}. \]
\end{enumerate}

\end{document}
